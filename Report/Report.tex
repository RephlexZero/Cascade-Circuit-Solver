\documentclass{article}
\usepackage[utf8]{inputenc}
\usepackage{geometry}
\usepackage{listings}
\usepackage{color}
\usepackage{graphicx}

\definecolor{codegreen}{rgb}{0,0.6,0}
\definecolor{codegray}{rgb}{0.5,0.5,0.5}
\definecolor{codepurple}{rgb}{0.58,0,0.82}
\definecolor{backcolour}{rgb}{0.95,0.95,0.92}

\lstdefinestyle{mystyle}{
    backgroundcolor=\color{backcolour},   
    commentstyle=\color{codegreen},
    keywordstyle=\color{magenta},
    numberstyle=\tiny\color{codegray},
    stringstyle=\color{codepurple},
    basicstyle=\footnotesize,
    breakatwhitespace=false,         
    breaklines=true,                 
    captionpos=b,                    
    keepspaces=true,                 
    numbers=left,                    
    numbersep=5pt,                  
    showspaces=false,                
    showstringspaces=false,
    showtabs=false,                  
    tabsize=2
}

\lstset{style=mystyle}

\title{Analysis and Implementation Report on Electrical Circuit Analysis Program}
\author{[Your Name]}
\date{[Date]}

\begin{document}

\maketitle

\section{Introduction}
This report outlines the development process of an electrical circuit analysis program, as specified 
in the structured programming coursework. It aims to detail the problem analysis, design decisions, 
and testing strategies that align with the provided marking rubric.

\section{Analysis of the Problem}
The main objective of the assignment is to create a program that analyzes 
electrical circuits using ABCD matrix analysis, with functionalities to read an 
input file, perform the analysis, and output the results in a specified format.

\subsection{Sub-tasks Identification}
\begin{itemize}
    \item Parsing the input file to extract circuit components and specifications.
    \item Implementing the ABCD matrix analysis for circuit analysis.
    \item Generating output files with the analysis results.
    \item Extending the program to include additional features such as exponent prefixes and decibel calculations.
\end{itemize}

\section{Design and Implementation}
This section breaks down the program into functions, classes, data structures, and arrays, detailing how each contributes to the overall solution.

\subsection{Outline of Functions and Classes}
The program is structured into several key components:
\begin{lstlisting}[language=Python]
# net_parser.py: Parses the input file and extracts circuit information.
# circuit.py: Contains classes and functions for circuit analysis.
# csv_writer.py: Manages the creation of the output file with analysis results.
# main.py: The main driver script that utilizes the above modules.
\end{lstlisting}

\subsubsection{Data Structures and Arrays}
The program utilizes lists and dictionaries to store circuit components and their connections, enabling efficient access and manipulation during analysis.

\section{Testing Strategy}
Comprehensive testing is vital to ensure the correctness and reliability of the program. 
The testing strategy covers unit tests for individual functions and integration tests for the overall program functionality.

\subsection{Function Testing}
Each function within the modules has associated tests to verify its expected behavior, particularly focusing on edge cases and error handling.

\subsection{Complete Program Testing}
Integration testing ensures that the program modules work together seamlessly, with tests designed to cover a variety of circuit configurations.

\section{Conclusion}
This report outlines the analytical approach and design considerations for the structured programming coursework on electrical circuit analysis.
 The detailed breakdown of functions, data structures, and testing strategies aims to provide clarity on the program's development process and adherence to the marking rubric.

\end{document}
