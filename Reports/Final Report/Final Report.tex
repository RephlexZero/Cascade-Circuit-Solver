\documentclass[conference]{IEEEtran}
\IEEEoverridecommandlockouts
\usepackage{cite}
\usepackage{amsmath,amssymb,amsfonts}
\usepackage{graphicx}
\usepackage{textcomp}
\usepackage{xcolor}
\usepackage{listings}
\usepackage{hyperref}

\definecolor{codegreen}{rgb}{0,0.6,0}
\definecolor{codegray}{rgb}{0.5,0.5,0.5}
\definecolor{codepurple}{rgb}{0.58,0,0.82}
\definecolor{backcolour}{rgb}{0.95,0.95,0.92}

\lstdefinestyle{mystyle}{
    backgroundcolor=\color{backcolour},   
    commentstyle=\color{codegreen},
    keywordstyle=\color{magenta},
    numberstyle=\tiny\color{codegray},
    stringstyle=\color{codepurple},
    basicstyle=\ttfamily\footnotesize,
    breakatwhitespace=false,         
    breaklines=true,                 
    captionpos=b,                    
    keepspaces=true,                 
    numbers=left,                    
    numbersep=5pt,                  
    showspaces=false,                
    showstringspaces=false,
    showtabs=false,                  
    tabsize=2
}

\lstset{style=mystyle}

\hypersetup{
    colorlinks=true,
    linkcolor=blue,
    filecolor=magenta,      
    urlcolor=cyan,
}

\begin{document}

\title{Cascade Circuit Analyser Final Report\\
{\footnotesize \textsuperscript{*}EE20084 - Structured Programming}
}

\author{\IEEEauthorblockN{Jake Stewart}
    \IEEEauthorblockA{\textit{Department of Electrical and Electronic Engineering} \\
        \textit{University of Bath}\\
        Bath, United Kingdom \\
        email: js3910@bath.ac.uk}
}
\maketitle

\begin{abstract}
    This report details the design choices, implementation and testing of the Cascade Circuit Analyser program.
    embedded.
\end{abstract}

\begin{IEEEkeywords}
    circuit analysis, Python, pytest, regex, software testing
\end{IEEEkeywords}

\tableofcontents

\section{Overview of approach}
The program largely suck to the design specification, moulding and deviating slightly for improvements in performance
or consiceness.

The program is split into four main files, \texttt{main.py}, \texttt{circuit.py}, \texttt{parse\_net.py} and \texttt{csv\_writer.py}.
\texttt{main.py} is the entry point of the program, and is responsible for parsing the command line arguments, reading the input file and writing the output file.
\texttt{circuit.py} contains the \texttt{Circuit} class, which is responsible for storing the circuit data and performing the analysis.
\texttt{parse\_net.py} contains the \texttt{parse\_netlist} function, which is responsible for parsing the netlist file and returning a list of components, a dictionary of
terminations, and a list of outputs.
\end{document}